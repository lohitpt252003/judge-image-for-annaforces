\documentclass{article}
\usepackage[a4paper, margin=1in]{geometry}
\usepackage{hyperref}
\usepackage{xcolor}
\usepackage{listings}
\usepackage{amssymb} % For the square symbol in the To-Do list
\usepackage{tabularx}
\usepackage{enumitem}

% Hyperlink setup
\hypersetup{
    colorlinks=true,
    linkcolor=blue,
    filecolor=magenta,
    urlcolor=cyan,
    pdftitle={Judge Image for Annaforces},
}

% Code listing style
\definecolor{codegreen}{rgb}{0,0.6,0}
\definecolor{codegray}{rgb}{0.5,0.5,0.5}
\definecolor{codepurple}{rgb}{0.58,0,0.82}
\definecolor{backcolour}{rgb}{0.95,0.95,0.95}

\lstdefinestyle{mystyle}{
    backgroundcolor=\color{backcolour},
    commentstyle=\color{codegreen},
    keywordstyle=\color{magenta},
    numberstyle=\tiny\color{codegray},
    stringstyle=\color{codepurple},
    basicstyle=\footnotesize\ttfamily,
    breakatwhitespace=false,
    breaklines=true,
    captionpos=b,
    keepspaces=true,
    numbers=none,
    numbersep=5pt,
    showspaces=false,
    showstringspaces=false,
    showtabs=false,
    tabsize=2
}
\lstset{style=mystyle}

% Custom JSON language definition for listings
\lstdefinelanguage{json}{
    basicstyle=\normalfont\ttfamily\footnotesize,
    string=[s]{"}{"},
    comment=[l]{:\ "},
    commentstyle=\color{black},
    morekeywords={true,false,null},
    keywordstyle=\color{blue},
    stringstyle=\color{codepurple},
    showstringspaces=false
}

% Title and author
\title{Judge Image for Annaforces}
\author{\href{https://opensource.org/licenses/MIT}{License: MIT}}
\date{}

\begin{document}

\maketitle

A secure and isolated environment for executing untrusted code, primarily for use in competitive programming platforms or online judges. It uses Docker to create a sandbox that restricts resource usage (CPU time, memory) and prevents malicious code from affecting the host system.

\section{Features}
\begin{itemize}
    \item \textbf{Language Support:} Executes code written in C, C++, and Python.
    \item \textbf{Resource Limiting:} Enforces time and memory limits on code execution.
    \item \textbf{Secure Sandboxing:} Uses Docker containers to isolate the execution environment.
    \item \textbf{Detailed Results:} Provides detailed information about the execution, including:
    \begin{itemize}
        \item Standard output and standard error.
        \item Execution time and memory usage.
        \item Status (Success, Compilation Error, Runtime Error, Time Limit Exceeded, Memory Limit Exceeded).
    \end{itemize}
\end{itemize}

\section{Getting Started}

\subsection{Prerequisites}
\begin{itemize}
    \item \textbf{Docker:} You must have Docker installed and running on your system.
\end{itemize}

\subsection{Installation}
\begin{enumerate}
    \item Clone the repository:
    \begin{lstlisting}[language=bash]
git clone https://github.com/lohitpt252003/judge-image-for-annaforces.git
    \end{lstlisting}
    \item Navigate to the project directory:
    \begin{lstlisting}[language=bash]
cd judge-image-for-annaforces
    \end{lstlisting}
\end{enumerate}

\section{Examples}

\subsection{Example 1: Successful Python Execution}
\begin{lstlisting}[language=Python]
from good_one import execute_code
import json

python_code = """
import sys
name = sys.stdin.readline()
print(f"Hello, {name.strip()}!")
"""

result = execute_code(
    language='python',
    code=python_code,
    stdin='World',
    time_limit_s=5,
    memory_limit_mb=128
)

print(json.dumps(result, indent=2))
\end{lstlisting}

\textbf{Output:}
\begin{lstlisting}[language=json]
{
  "stdout": "Hello, World!",
  "stderr": "",
  "err": "",
  "timetaken": 0.0,
  "memorytaken": 3.68359375,
  "success": true
}
\end{lstlisting}

\subsection{Example 2: C++ Code with a Runtime Error}
\begin{lstlisting}[language=Python]
from good_one import execute_code
import json

cpp_code = """
#include <iostream>
#include <vector>
int main() {
    std::vector<int> v;
    std::cout << v.at(10); // This will throw an exception
    return 0;
}
"""

result = execute_code(language='c++', code=cpp_code, stdin='', time_limit_s=5, memory_limit_mb=128)

print(json.dumps(result, indent=2))
\end{lstlisting}

\textbf{Output:}
\begin{lstlisting}[language=json]
{
  "stdout": "",
  "stderr": "",
  "err": "Runtime Error (Exit Code: 134)",
  "timetaken": 0.0,
  "memorytaken": 0.0,
  "success": false
}
\end{lstlisting}

\section{API Reference}

\subsection{\texttt{execute\_code()}}
\begin{table}[h!]
    \centering
    \begin{tabularx}{\textwidth}{|l|X|l|}
        \hline
        \textbf{Argument} & \textbf{Description} & \textbf{Default Value} \\
        \hline
        \texttt{language} & The programming language (`c`, `c++`, `python`). & \texttt{'python'} \\
        \hline
        \texttt{code} & The source code to execute. & \texttt{"print(\dots{})"} \\
        \hline
        \texttt{stdin} & The standard input for the code. & \texttt{''} \\
        \hline
        \texttt{time\_limit\_s} & The time limit in seconds. & \texttt{2} \\
        \hline
        \texttt{memory\_limit\_mb} & The memory limit in megabytes. & \texttt{1024} \\
        \hline
    \end{tabularx}
\end{table}

\textbf{Returns:} A dictionary containing the execution results with the following keys:
\begin{itemize}
    \item \texttt{stdout} (str): The standard output of the code.
    \item \texttt{stderr} (str): The standard error of the code.
    \item \texttt{err} (str): An error message if an error occurred.
    \item \texttt{timetaken} (float): The time taken for the code to execute in seconds.
    \item \texttt{memorytaken} (float): The memory used by the code in megabytes.
    \item \texttt{success} (bool): Whether the code executed successfully.
\end{itemize}

\section{How It Works}
The core logic is in the \texttt{execute\_code} function within \texttt{good\_one.py}. Here's a breakdown of the process:
\begin{enumerate}
    \item \textbf{Input Validation:} Checks if the requested programming language is supported.
    \item \textbf{Docker Setup:}
    \begin{itemize}
        \item Verifies that Docker is running.
        \item Checks if the required Docker image (\texttt{sandbox-image:latest}) exists.
        \item If the image is not found, it builds it dynamically from a simple \texttt{Dockerfile} definition.
    \end{itemize}
    \item \textbf{File Preparation:}
    \begin{itemize}
        \item Creates a temporary directory on the host machine.
        \item Saves the user's source code and standard input to files within this directory.
    \end{itemize}
    \item \textbf{Container Management:}
    \begin{itemize}
        \item Starts a detached Docker container from the \texttt{sandbox-image}.
        \item Copies the source code and input files into the container.
    \end{itemize}
    \item \textbf{Code Compilation (for C/C++):}
    \begin{itemize}
        \item If the language is C or C++, it compiles the code inside the container using \texttt{gcc} or \texttt{g++}.
        \item If compilation fails, it returns a "Compilation Error."
    \end{itemize}
    \item \textbf{Code Execution:}
    \begin{itemize}
        \item Executes the compiled binary (for C/C++) or the Python script.
        \item The execution is wrapped with \texttt{/usr/bin/time -v} to measure resource usage and \texttt{timeout} to enforce the time limit.
    \end{itemize}
    \item \textbf{Result Parsing:}
    \begin{itemize}
        \item Captures the standard output, standard error, and exit code of the process.
        \item Parses the output of \texttt{/usr/bin/time -v} to extract the execution time and memory consumption.
        \item Determines the final status based on the exit code (e.g., exit code 124 indicates a timeout).
    \end{itemize}
    \item \textbf{Cleanup:}
    \begin{itemize}
        \item Stops and removes the Docker container.
        \item Deletes the temporary directory from the host.
    \end{itemize}
\end{enumerate}

\section{To-Do / Improvements}
\begin{itemize}[label={$\square$}]
    \item Add support for more languages (e.g., Java, Go, Rust).
    \item Implement a more robust queueing system for handling multiple submissions.
    \item Enhance security by running the code as a non-root user inside the container.
    \item Improve the accuracy of resource measurement.
    \item Create a REST API wrapper around the execution logic.
    \item Add more comprehensive unit and integration tests.
    \item Parameterize the Docker image name and other constants.
\end{itemize}

\section{Contributing}
Contributions are welcome! Please feel free to submit a pull request.

\section{Authors}
\begin{itemize}
    \item \href{https://github.com/lohitpt252003}{Lohit P Talavar}
    \item \href{https://github.com/adarshmishra121}{Adarsh Mishra}
    \item \href{https://github.com/MeenaPriyanshi}{Priyanshi Meena}
\end{itemize}

\section{License}
This project is licensed under the MIT License. See the LICENSE file for details.

\section{Credits}
\begin{itemize}
    \item Python
    \item Docker
    \item Linux (Ubuntu)
\end{itemize}

\end{document}