\documentclass{article}
\usepackage{hyperref}
\usepackage{longtable}
\begin{document}

\title{judge-image-for-annaforces}
\author{Lohit P Talavar, Adarsh Mishra, Priyanshi Meena}
\date{}
\maketitle

This project provides a Python script that allows you to compile and execute code (Python, C, C++) securely in a Docker container, enforcing time and memory limits. It is primarily designed for automated code judging, competitive programming, or sandboxed code evaluation environments.

\section*{Features}
\begin{itemize}
    \item \textbf{Language support}: Python, C, C++
    \item \textbf{Security}: Runs code inside a Docker container
    \item \textbf{Resource limits}: Enforces memory and time limits for execution
    \item \textbf{Automatic Docker image management}: Builds the judge image if not present
    \item \textbf{Logging}: Outputs events and errors to \texttt{judge\_runner.log}
    \item \textbf{Easy file management}: Functions to create, delete, and manage test files
    \item \textbf{Static Safety Checks}: Blocks forbidden imports and system calls before execution
\end{itemize}

\section*{Forbidden Keywords and Headers}

To ensure security and sandboxing, the following imports, includes, and functions are \textbf{blocked} and will prevent code execution if found:

\subsection*{Python}
\begin{itemize}
    \item \texttt{import os}
    \item \texttt{import subprocess}
    \item \texttt{import shutil}
    \item \texttt{import socket}
    \item \texttt{import ctypes}
    \item \texttt{import pathlib}
    \item \texttt{from os}
    \item \texttt{open(}
\end{itemize}

\subsection*{C}
\begin{itemize}
    \item \texttt{\#include <unistd.h>}
    \item \texttt{\#include <sys/} (e.g., sys/socket.h, sys/wait.h)
    \item \texttt{\#include <dlfcn.h>}
    \item \texttt{system(}
    \item \texttt{popen(}
    \item \texttt{fork(}
    \item \texttt{exec}
\end{itemize}

\subsection*{C++}
\begin{itemize}
    \item \texttt{\#include <unistd.h>}
    \item \texttt{\#include <sys/}
    \item \texttt{\#include <dlfcn.h>}
    \item \texttt{system(}
    \item \texttt{popen(}
    \item \texttt{fork(}
    \item \texttt{exec}
    \item \texttt{\#include <filesystem>}
\end{itemize}

If any of these patterns are detected in your code, execution will be blocked for safety reasons.

\section*{Defaults}
\begin{longtable}{|p{4cm}|p{5cm}|p{6cm}|}
\hline
\textbf{Parameter} & \textbf{Default Value} & \textbf{Description} \\
\hline
Docker Image Name & \texttt{judge-image} & The name of the Docker image \\
Docker Image Tag & \texttt{latest} & The tag for the Docker image \\
Full Docker Image & \texttt{judge-image:latest} & Image name used for execution \\
Test Folder & \texttt{test} & Folder in which files are created \\
Test File & \texttt{main.py} & Default filename for Python \\
Default Compile Time Limit & 5 seconds & Time allowed for compiling C/C++ \\
Default Run Time Limit & 1 second & Time allowed for running code \\
Default Memory Limit & 1024 MB & Max memory allowed in container \\
Logging File & \texttt{judge\_runner.log} & Log file for execution events/errors \\
\hline
\end{longtable}

\section*{Usage}

\subsection*{1. Prerequisites}
\begin{itemize}
    \item Docker must be installed and running on your system.
    \item Python 3.x
\end{itemize}

\subsection*{2. Quick Start}
Clone the repository and run the script:
\begin{verbatim}
python app.py
\end{verbatim}

\subsection*{3. Main Functions}

\paragraph{create\_folder\_and\_file(folder\_name, file\_name, content)}
Creates a folder and writes the specified content to a file. \\
\textbf{Defaults:}
\begin{itemize}
    \item folder\_name: \texttt{test}
    \item file\_name: \texttt{main.py}
    \item content: Minimal Python print statement
\end{itemize}

\paragraph{delete\_folder(folder\_name)}
Deletes the folder (and its contents) specified.\\
\textbf{Default}: \texttt{test}

\paragraph{create\_image(image)}
Checks if the Docker image exists. If not, builds the image from the current directory.\\
\textbf{Default}: \texttt{judge-image:latest}

\paragraph{run\_code\_in\_container(image, file\_path, language, stdin, time\_limit, memory\_limit)}
\begin{itemize}
    \item \textbf{image}: Docker image to use (default: \texttt{judge-image:latest})
    \item \textbf{file\_path}: Path to the source code file (default: \texttt{test/main.py})
    \item \textbf{language}: \texttt{'python'}, \texttt{'c'}, or \texttt{'c++'} (default: \texttt{'python'})
    \item \textbf{stdin}: Input to provide to the program (default: \texttt{''})
    \item \textbf{time\_limit}: Seconds allowed for execution (default: 1)
    \item \textbf{memory\_limit}: Memory limit in MB (default: 1024)
\end{itemize}

Returns a result dictionary:
\begin{verbatim}
{
    'success': True/False,
    'stdout': 'Program output',
    'stderr': 'Error output',
    'error': 'Error message if any'
}
\end{verbatim}

\subsection*{4. Example}
\begin{verbatim}
from app import create_folder_and_file, create_image, run_code_in_container, delete_folder

cpp_code = (
    '#include <iostream>\n'
    'using namespace std;\n'
    'int main() {\n'
    '    string s; cin >> s;\n'
    '    cout << "Echo: " << s << endl;\n'
    '    return 0;\n'
    '}\n'
)
create_folder_and_file(file_name='main.cpp', content=cpp_code)
create_image()
result = run_code_in_container(language='c++', file_path='test/main.cpp', stdin='HelloWorld')
print(result)
delete_folder()
\end{verbatim}

\subsection*{5. Logging}
All execution logs, errors, and build information are written to \texttt{judge\_runner.log} in append mode.

\subsection*{6. Troubleshooting}
\begin{itemize}
    \item Ensure Docker is running and you have permissions to execute containers.
    \item If you get permission errors, check file ownership and Docker setup.
    \item The Docker image must support the target language (update Dockerfile if needed).
\end{itemize}

\subsection*{7. Extending}
\begin{itemize}
    \item To add more languages, update the command generation logic in \texttt{run\_code\_in\_container}.
    \item To change resource limits, adjust the \texttt{--memory} flag and \texttt{timeout} parameters.
\end{itemize}

\section*{License}
MIT License

\section*{Authors}
\begin{itemize}
    \item \href{https://github.com/lohitpt252003}{Lohit P Talavar}
    \item \href{https://github.com/adarshmishra121}{Adarsh Mishra}
    \item \href{https://github.com/MeenaPriyanshi}{Priyanshi Meena}
\end{itemize}

\section*{Credits}
\begin{itemize}
    \item Uses Docker for sandboxing.
    \item Python standard libraries for subprocess and file management.
\end{itemize}

\end{document}